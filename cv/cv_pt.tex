%! TEX program = xelatex
\documentclass[a4paper,10pt]{article}

%A Few Useful Packages
\usepackage{marvosym}
\usepackage{fontspec} 					%for loading fonts
\usepackage{xunicode,xltxtra,url,parskip} 	%other packages for formatting
\RequirePackage{color,graphicx}
\usepackage[usenames,dvipsnames]{xcolor}
\usepackage[big]{layaureo} 				%better formatting of the A4 page
% an alternative to Layaureo can be ** \usepackage{fullpage} **
\usepackage{supertabular} 				%for Grades
\usepackage{titlesec}					%custom \section

%Setup hyperref package, and colours for links
\usepackage{hyperref}
\definecolor{linkcolour}{rgb}{0,0.2,0.6}
\hypersetup{colorlinks,breaklinks,urlcolor=linkcolour, linkcolor=linkcolour}

%FONTS
\defaultfontfeatures{Mapping=tex-text}
%\setmainfont[SmallCapsFont = Fontin SmallCaps]{Fontin}
%%% modified for Karol Kozioł for ShareLaTeX use
\setmainfont[
SmallCapsFont = Fontin-SmallCaps.otf,
BoldFont = Fontin-Bold.otf,
ItalicFont = Fontin-Italic.otf
]
{Fontin.otf}
%%%

%CV Sections inspired by: 
%http://stefano.italians.nl/archives/26
\titleformat{\section}{\Large\scshape\raggedright}{}{0em}{}[\titlerule]
\titlespacing{\section}{0pt}{3pt}{3pt}
%Tweak a bit the top margin
%\addtolength{\voffset}{-1.3cm}

%Italian hyphenation for the word: ''corporations''
\hyphenation{im-pre-se}

%-------------WATERMARK TEST [**not part of a CV**]---------------
\usepackage[absolute]{textpos}

\setlength{\TPHorizModule}{30mm}
\setlength{\TPVertModule}{\TPHorizModule}
\textblockorigin{2mm}{0.65\paperheight}
\setlength{\parindent}{0pt}

%--------------------BEGIN DOCUMENT----------------------
\begin{document}

%WATERMARK TEST [**not part of a CV**]---------------
%\font\wm=''Baskerville:color=787878'' at 8pt
%\font\wmweb=''Baskerville:color=FF1493'' at 8pt
%{\wm 
%	\begin{textblock}{1}(0,0)
%		\rotatebox{-90}{\parbox{500mm}{
%			Typeset by Alessandro Plasmati with \XeTeX\  \today\ for 
%			{\wmweb \href{http://www.aleplasmati.comuv.com}{aleplasmati.comuv.com}}
%		}
%	}
%	\end{textblock}
%}

\pagestyle{empty} % non-numbered pages

\font\fb=''[cmr10]'' %for use with \LaTeX command

%--------------------TITLE-------------
\par{\centering
		{\Huge Guilherme Smethurst Albuquerque
	}\bigskip\par}

%--------------------SECTIONS-----------------------------------
%Section: Personal Data
\section{Dados Pessoais}

\begin{tabular}{rl}
    \textsc{Lugar e ano de nascimento:} & Recife-PE, Brasil  | 1997 \\
    \textsc{email:}     &  \href{mailto:guilherme.smta@gmail.com}{guilherme.smta@gmail.com}
\end{tabular}

\section{Objetivo}
Área de desenvolvimento de software, mais especificamente desenvolvimento backend

%Section: Work Experience at the top
\section{Experiências de Trabalho}
\begin{tabular}{r|p{11cm}}
\emph{Dez 2021 - Now} & Desenvolvedor Node.js em FanHero
 \\ & \footnotesize{Trabalhando na reengenharia de um sistema complexo de OTT, video sob demanda, gerenciamento de conteudo e mais (veja mais em \href{https://fanhero.com}{FanHero}). Stack baseada em Node.js/NestJS/GraphQL/REsT/Redis/MongoDB}\\\\
\emph{Jul 2021 - Now} & Desenvolvedor Node.js em Compasso UOL
 \\ & \footnotesize{Trabalhando na manutenção e novas features para microsserviços em Node.js, Express.js, bem como novas features em um gateway de serviços (Node.js/Express.js/Apache Kafka/Docker/Jenkins}\\\\
\emph{Mar 2021} & Desenvolvedor Salesforce em Top Information
 \\ & \footnotesize{Trabalhando para manutenção e otimização de sistemas, bem como reparos em bugs de regras de negócio. Também fazendo manutenção do sistema BSC - PwC (Vue.js + Adonis.js) como um trabalho freelancer}\\\\
 \emph{Mar 2020} & Desenvolvedor Node.js em Abacomm
 \\ & \footnotesize{Entregou aplicações baseadas na stack AWS Lambda+DynamoDB, com foco em segurança e otimização de queries} \\
 & \footnotesize{Trabalhou como desenvolvedor fullstack (Vue.js/AdonisJS) em sistema corporativo para a empresa de auditoria Pricewaterhouse Coopers (PwC), com foco na integracao entre API e front-end, fluxo de dados da aplicação, bem como boas práticas de UX com formulários e validação} \\\\
 \emph{Aug 2019} & Desenvolvedor Fullstack no Departamento de Produtos Informacionais da Biblioteca Central UFPA\\ & \footnotesize{Trabalhou na reegenharia do sistema FICAT. Trata-se de um sistema de geração automática de fichas catalográficas e conta com os módulos de geração das fichas em PDF, estatísticas e gráficos sobre a geração das fichas (com Chart.js) e está sendo desenvolvida em ambiente Node.js (com Koa.js) e frontend Vue.js, embarcados em um projeto com Nuxt.js.} \\\\
 \textsc{Aug 2018} & Desenvolvimento de soluções de \textit{software} para melhoria de processos de \textit{software}\\& \footnotesize{Atualmente refatorando e desenvolvendo código-fonte para o sistema SPM (Software Process Marketplace)}\\\\
 \textsc{Jun 2017} & Desenvolvedor em Laboratório de Engenharia de \textit{software} - Universidade Federal do Pará \\ & \footnotesize{Desenvolveu soluções em código-fonte para o módulo de Gerência de Configuração de Software para o sistema WebAPSEE}\\\multicolumn{2}{c}{} \\
\end{tabular}

\section{Experiências de Pesquisa Acadêmica}
\begin{tabular}{r|p{11cm}}
 \emph{2018} & Campos, U. F., Albuquerque S. G. et al (2018)\\
 & \textit{Mining Rule Violations in JavaScript Code Snippets}\\
 & MSR 2019 Mining Challenge\\\multicolumn{2}{c}{} \\
\end{tabular}

%Section: Education
\section{Formação Acadêmica}
\begin{tabular}{rl}	
 \textsc{Maio} 2016-Dezembro 2018 & Graduando em \textsc{Ciência da Computação}, \\
& \textbf{Universidade Federal do Pará}, Belém-PA, Brasil
\end{tabular}

%Section: Scholarships and additional info
\section{Formação Complementar}
\begin{tabular}{rl}
\textsc{Dezembro} 2017 & Curso de Algoritmos Inteligentes de Busca\\ & (\textsc{Udemy}, \footnotesize{currículo disponível \href{https://www.udemy.com/certificate/UC-9414I0UR/}{neste link}})\\
\textsc{Julho} 2019 &  Construindo Aplicações Web Com o Novo Angular (4, 5 e 6)\\ & (\textsc{Udemy}, \footnotesize{currículo disponível \href{https://www.udemy.com/certificate/UC-P6VGWTET/}{neste link}})
\end{tabular}

%Section: Languages
\section{Linguagens}
\begin{tabular}{rl}
 \textsc{Português Brasileiro:} & Língua materna \\
\textsc{Inglês:} & Nível equivalente a B1-B2 em inglês geral (nenhum certificado disponível)\\
\textsc{Espanhol:} & Lê e escreve pouco \\
\end{tabular}

\section{Habilidades}
\begin{tabular}{rl}
 Conhecimento Avançado:& Linux, Shellscripting, AdonisJS, NestJS\\& JavaScript, TypeScript, Vue.js, Node.js, Vue Router, Vuex, \\&DynamoDB, AWS Lambda, GraphQL, Serverless Framework\\\\
 Conhecimento Intermediário:& \textsc{C}, Python\\ &HTML, CSS, My\textsc{SQL}, Bootstrap, Bulma,\\& DOM, Lucid ORM\\& Nuxt.js, Koa.js, Bookshelf.js, JWT Auth,\\&Angular 4+, MongoDB, Docker\\\\
 Conhecimento básico: & PostgreSQL, Knex, Electron, Chai, \\ & Electron, Nginx, Redis, Apache Kafka, C++,  {\fb\LaTeX}\setmainfont[SmallCapsFont=Fontin-SmallCaps.otf]{Fontin.otf},\\
 &React.js, Redux, Jenkins
\end{tabular}

\section{Links Úteis}
\href{https://github.com/Lakshamana}{GitHub}, \href{https://www.linkedin.com/in/guilherme-albuquerque-622421127}{LinkedIn}\\ 

\section{Interesses e Atividades}
Tecnology, Open-Source Programming, Desktop e Web, \\Software Engineering, Fullstack Development, Productivity, Vim/Terminal lifestyle

%\XeTeXpdffile ''GMAT.pdf'' page 1 scaled 800
\end{document}/
